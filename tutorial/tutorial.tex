
\documentclass{article}

\begin{document}

\title{Tutorial for Phase Estimation}
\author{Rhys Hawkins}
\maketitle

\section{Overview}

This tutorial provides a worked example for estimating phase and group dispersion
from ambient noise cross-correlations using the method described in

\begin{flushleft}
  Hawkins R. and Sambridge M., ``An adjoint technique for estimation of interstation phase and
  group dispersion from ambient noise cross-correlations``, BSSA, 2019
\end{flushleft}

Within this directory, there are bash scripts to run an example inversion of a ambient noise
cross-correlation to obtain path average dispersion/models. The main steps for this inversions
are

\begin{enumerate}
\item Create a reference model and dispersion curve
\item Estimate points along the phase velocity dispersion curve from peaks, troughs, and zero-crossings
  of the Noise Correlation Function.
\item Fit a 1D Path Average Earth model to the trial dispersion curves
\item Fit a 1D Path Average Earth model and amplitude model to the Noise Correlation function
\end{enumerate}

\subsection{Quick run}

Each of these steps has a corresponding bash script for running the various stages of
the inversion process, to perform an inversion of one of the example noise correlation functions,
from within this directory, once the various prerequisites are installed and programs within
this software package are compiled, you can run in order:

\begin{verbatim}
> 00_create_reference.sh
> 01_create_initial_target_phase.sh
> 02_fit_initial_target_phase.sh
> 03_fit_bessel.sh
> python2 scripts/plot_bessel_result.py -f Final_HOT05_HOT25
\end{verbatim}

The first of these will generally only need to be run once, whereas the final three
are required for each station pair. These final three steps take approximately
6 minutes to run on a laptop computer. These scripts run command line programs which
are explained in further detail in the following sections.

\section{Creating a reference model}

A reference model is required as a starting point for this inversion process. The file format used
to describe the reference model is a simple text file, a simple example

\begin{verbatim}
2
10.0 0 3.50
0.0 0 4.80
\end{verbatim}

\section{Creating initial phase dispersion curves}

This inversion methodology uses an extension to previous methods for zero-crossing to include peaks and
troughs in addition to treating Love and Rayleigh observations jointly. This phase of the inversion is
implemented in a Python script {\texttt InitialPhase/scripts/estimate\_joint\_phase\_amplitude.py}. This script
assumes that the input files are arranged into directories and files as follows

\begin{verbatim}
BaseDirectory
  |
  |- LoveResponse
  |  |
  |  |- dispersion_<stationA>_<stationB>.txt
  |  |- ...
  |
  |- RayleighResponse
     |
     |- dispersion_<stationA>_<stationB>.txt
     |- ...
\end{verbatim}

This layout is used in the {\texttt example\_data} directory. This directory layout facilitates running the
{\texttt InitialPhase/scripts/estimate\_joint\_phase\_amplitude.py} script by specifying the base path
and a station pair to process. For example

\begin{verbatim}
python2 ../InitialPhase/scripts/estimate_joint_phase_amplitude.py \
    -p ../example_data -s HOT05_HOT25 
\end{verbatim}

This will predict an initial phase dispersion curve for Love and Rayleigh dispersion based on picking
the peaks, troughs and zero crossings of the noise correlation function. As run above, this script
uses by default the reference models in the tutorial subdirectory and plots the result of the picking
and the phase dispersion curves for Love and Rayleigh waves. The important options for this script
are

\begin{description}
\item[-p|--path] The base path for the dispresion data files
\item[-s|--station-pair] The station pair, eg HOT05\_HOT25, to invert
\item[-f|--freq-min] Minimum frequency (Hz)
\item[-F|--freq-max] Maximum frequency (Hz)
\item[--filter] Gaussian filter width for smoothing Bessel functions (0 = no filtering)
\item[--noshow] Don't show the plots (useful for background processing)
\item[-o|--output] Output the trial dispersion curves to files (write two files with .love and .rayleigh suffixes)
\end{description}

The frequency min and max values specify the range of frequencies to
consider when searching for peaks, troughs, and zero-crossings
(default is 0.025 to 0.35 Hz). A tunable parameter is the filter width
that can be used to smooth NCF Bessel functions, values between 0 and
3 produce consistent results but this may need to be adjusted. By
default, the script simply shows a plot but with the noshow and output
options, the trial dispersion curves for Love and Rayleigh waves can
be written to files without plot output which is useful for bulk
processing of data.

\section{Fitting an Earth model to trial dispersion curves}

The previous step generates a proposed phase velocity dispersion curves for Love and Rayleigh
waves. These will naturally have noise and frequency gaps, so a first step is to fit a physically
based dispersion curve to these trial curves to overcome both these issues.

The program that performs this step is {\texttt InitialPhase/optimizer/optimizejoint} which is
a C++ program that uses a gradient descent approach to converge a starting model (reference model)
to the trial dispersion curves. The important options for this program are

\begin{description}
\item [-i|--input-love] Input dispersion data file for Love waves
\item [-I|--input-rayleigh] Input dispersion data file for Rayleigh waves
\item [-c|--phase-love] Initial Love dispersion curve to fit
\item [-C|--phase-rayleigh] Initial Rayleigh dispersion curve to fit
\item [-r|--reference] Reference model file
\item [-R|--sigma-rho] Gaussian prior width on density (kg/m$^3$)
\item [-V|--sigma-vs] Gaussian prior width on $V_s$ (m/s)
\item [-X|--sigma-xi] Gaussian prior width on $\xi$
\item [-S|--sigma-vpvs] Guassian prior width on $V_p$/$V_s$ ratio
\item [-N|--nsteps] Number of iterations to run
\item [-T|--thin] Number of frequency bins to thin to approximate forward modelling
\item [-o|--output] Output prefix for various files written on successful completion
\end{description}

The first five command line parameters are for input data files
representing the observed noise correlation function, the trial
dispersion curves to fit, and the starting or reference model.

The next four parameters specify the prior widths for the four model
parameters used for the 1D path average Earth model. We've generally
found that relatively wide prior widths of 500 kg/m$^3$ for density,
500 m/s for $V_s$, and 0.05 for $\xi$ and $V_p$/$V_s$ ratio.

The number of steps to run is a trade-off between running time and
convergence. Generally we use between 30 - 50 iterations. If a large
number of iterations is specified, the program will terminate when the
gradient descent approach cannot improve the model further.

The thinning option allows an approximate forward model where the full
adjoint is only computed at every $n$th frequency bin based on the
thinning parameter. This is achieved by using a piece-wise Hermite
interpolant between points for the wavenumber $k$ and its deriviatives
with respect to model parameters as a function of frequency.  This
allows a large increase in speed of computation at a relative low cost
in terms of accuracy. Values of around 20 start to incur errors around
1\%, so a recommendation is a value of around 15 is acceptable.

Lastly the output option specifies an output prefix for multiple output files
that are written upon successul completion of the optimizaiton. These files
are

\begin{description}
\item[*.model] The final model of the optimization (used as input for the next step)
\item[*.pred-love] The final predictions of Love wave dispersion
\item[*.pred-rayleigh] The final predictions of Rayleigh wave dispersion
\end{description}

The script {\texttt 02\_fit\_initial\_target\_phase.txt} provides an example
of running this phase of the optimization approach.


\section{Fitting Bessel Functions}

The final step uses the final model obtained from the previous step to
fit the envelope and Bessel function of the NCF. The important command line options
are similar to the previous step and we outline them below

\begin{description}
\item [-i|--input-love] Input dispersion data file for Love waves
\item [-I|--input-rayleigh] Input dispersion data file for Rayleigh waves
\item [-r|--reference] Reference/starting model file (from previous step)
\item [-R|--sigma-rho] Gaussian prior width on density (kg/m$^3$)
\item [-V|--sigma-vs] Gaussian prior width on $V_s$ (m/s)
\item [-X|--sigma-xi] Gaussian prior width on $\xi$
\item [-S|--sigma-vpvs] Guassian prior width on $V_p$/$V_s$ ratio
\item [-N|--nsteps] Number of iterations to run
\item [-T|--thin] Number of frequency bins to thin to approximate forward modelling
\item [-G|--gaussian-smooth] Width of Gaussian filter to Smooth the envelope of the NCF (we often use 0 (no smoothing) or 1)
\item [-J|--jacobians] Output Jacobians (for uncertainty estimates)
\item [-o|--output] Output prefix for various files written on successful completion
\end{description}

The first two parameters are the input dispersion data files as in the
previous step. The reference parameter should point to the output
model file from the previous step.

The next four parameters are prior widths as before.

The nsteps and thin parameters are as in the previous step.

We experimented with Gaussian smoothing of the envelope of the NCF and
it has little effect. We generally use 0 (no-smoothing) or 1.

The Jacobians option computes and outputs extra files representing
model covariance and Jacobians of the forward model which can be
post-processed to approximate uncertainties.

Lastly, the output parameter specifies the output prefix for writing
the output files on successful completion of the optimization.

\begin{description}
\item[*.initpred-love] Initial Love wave predictions
\item[*.initpred-rayleigh] Initial Rayleigh wave predictions
\item[*.pred-love] Final Love wave predictions
\item[*.pred-rayleigh] Final Rayleigh wave predictions
\item[*.model] Final model
\end{description}

If the Jacobians option is specified, the following extra files are
written:

\begin{description}
\item[*.Cm] Model covariance matrix
\item[*.love\_Cd] Love observation diagonal of data covariance matrix
\item[*.rayleigh\_Cd] Rayleigh observation diagonal of data covariance matrix
\item[*.love\_G] Love forward model matrix
\item[*.love\_Jc] Love Jacobians for phase velocity
\item[*.love\_JU] Love Jacobians for group velocity
\item[*.rayleigh\_G]  Rayleigh forward model matrix
\item[*.rayleigh\_Jc] Rayleigh Jacobians for phase velocity
\item[*.rayleigh\_JU] Rayleigh Jacobians for group velocity
\end{description}  

An example of running this program is given in the {\texttt 03\_fit\_bessel.sh} script.

\section{Analysing results}

In the {\texttt tutorial/scripts} directory there are some python scripts
(requiring numpy and matplotlib packages) for plot the results. These also
provide examples of how to extract information from the files output
during the final stage of inversion.

\subsection{Verifying Fit}

To plot the final fit of the predicted Bessel functons to the
observered real spectrum of the NCF:

\begin{verbatim}
python2 scripts/plot_bessel_result.py -f Final_HOT05_HOT25 -d ../example_data
\end{verbatim}

The real data is shown in black, with the fitted Bessel shown in red.

\subsection{Comparing Group Velocities}

To plot the predicted group velocities versus the FTAN images:

\begin{verbatim}
python2 scripts/plot_group_result.py -f Final_HOT05_HOT25 -d ../example_data
\end{verbatim}

The FTAN analysis is shown shaded in gray scale (try --cm Blues for
colour) with the predicted group velocity curve plotted with a black
line.

\subsection{Uncertainties}

To plot 95\% confidence intervals for Group and Phase velocity
estimates:

\begin{verbatim}
python2 scripts/plot_uncertainties.py -f Final_HOT05_HOT25
\end{verbatim}

This uses an eigen decomposition to project the covariance matrix
(hyper-ellipse) into phase or group velocity confidence intervals.

\subsection{Inverting Rayleigh data only}

It is possible to invert only Rayleigh wave data in the case where only vertical
data is available (for example). Addition scripts are provided with the post
fix ``rayleigh'' analogous to the scripts described above. In summary,
once a reference model is created, inverting the example data for Rayleigh waves
only can be achieved using

\begin{verbatim}
> 01_create_initial_target_phase_rayleigh.sh
> 02_fit_initial_target_phase_rayleigh.sh
> 03_fit_bessel_rayleigh.sh
> python2 scripts/plot_bessel_result_rayleigh.py -f Final_HOT05_HOT25
\end{verbatim}

Most of the command line arguments within these scripts have obvious counterparts
to those described above.

\appendix
  
\section{Input File Format}

The file format used by the programs here for observed ambient noise
cross-correlations is a simple text file format with some meta
information on the first two lines, followed by the noise correlation
function and empirical Greens function in the frequency domain.  An
example for the first 4 lines of one of the example data files is
shown below (these files are located in the example\_data
subdirectory).

\begin{verbatim}
  -21.09639    65.10969   -16.65020    65.05438   209.15596
  2.000000 197 11.33283 6.21773 4097
  0.000000 2.14409e-02  0.00000e+00  1.94272e-03  0.00000e+00
  0.000244 2.13633e-02 -2.33984e-06  1.33636e-02  7.39238e-04
\end{verbatim}

The first line contains longitude and latitude of the two stations,
i.e. <lonA> <latA> <lonB> <latB>, followed by the interstation
distance in kilometres. The second line contains the sampling rate of
the signal in Hertz, the number of days of cross-correlations, the
signal to noise ratios for the acausal and causal signals, and finally
the number of points in the frequency domain representation of the
noise correlation functions.

The remaining lines contains the frequency domain representations of
the empirical Greens functions and noise correlation functions where
each line consists of <frequency>, <real EGF> <imaginary EGF> <real
NCF> <imaginary NCF>. The frequency domain representation is assumed
to be of a real signal and hence only the positive frequencies are
required.

For this inversion approach, the important parts of the data file
required are the interstation distance, the sampling rate and the
first and forth column (frequency and real component of the NCF). The
EGF columns are only used for comparing Group velocity predictions.


\end{document}

